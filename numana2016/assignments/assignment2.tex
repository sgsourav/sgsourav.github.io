\documentclass[11pt,a4paper]{article}
\usepackage[vmargin=1.5in, hmargin=1in]{geometry}

\usepackage{amsmath,amsfonts,amsthm}

\usepackage[T1]{fontenc}
\usepackage{enumerate}

\setlength{\parskip}{5pt}
\setlength{\parindent}{0pt}

\usepackage{fancyhdr}
\pagestyle{fancy}
\fancyhf{}
\rhead{\textsf{BStat--I, ISI Kolkata}}
\lhead{\textsf{Numerical Analysis}}
\rfoot{\textsf{Page \thepage}}

\newcommand{\R}{\textsf{R}}

\begin{document}

\begin{center}
  
  \textbf{\Huge Assignment 2}

  \vspace*{10pt}  

  \textbf{Posted on 12 Feb 2016 $ \ \vert \ $ Clarify doubts by 16 Feb 2016 $ \ \vert \ $ Submit by 19 Feb 2016}

  \vspace*{10pt}  

  This is a \emph{Group Assignment} -- each group (two students) should submit a single set of solutions. 

  \vspace*{10pt}  
  
  The solutions may be submitted either as a clearly legible hand-written document, or as a single \LaTeX\ generated PDF document. In case of PDF submission, the filename should be \texttt{assign2\_groupXX.pdf}, where \texttt{XX} is the serial number of the group. Be cogent, but concise.
    
  \vspace*{10pt}  

  \textbf{Attempt all problems. This assignment is worth 50 points in total.}

  \vspace*{20pt}

  \hrule

  \vspace*{10pt}
   
\end{center}

\textbf{Background:} If $X \in \mathbb{R}^{n \times n}$ is nonsingular, then the map $A \mapsto X^{-1}AX$ for $A \in \mathbb{R}^{n \times n}$ is called a \emph{similarity transformation} of $A$. Two matrices $A$ and $B$ are \emph{similar} if there is a similarity transformation relating one to the other, i.e., if there exists a nonsingular $X \in \mathbb{R}^{n \times n}$ such that $B = X^{-1}AX$. If $Q \in \mathbb{R}^{n \times n}$ is orthogonal, i.e., if $QQ^{T} = Q^{T}Q = I$, then the map $A \mapsto Q^{T}AQ$ for $A \in \mathbb{R}^{n \times n}$ is called a \emph{unitary similarity transformation} of $A$.

A square matrix $A \in \mathbb{R}^{n \times n}$ is called \emph{diagonalizable} if it is similar to a diagonal matrix, i.e., if there exists a nonsingular $X \in \mathbb{R}^{n \times n}$ such that $D = X^{-1}AX$ is a diagonal matrix. A square matrix $A \in \mathbb{R}^{n \times n}$ is called \emph{unitarily diagonalizable} if there exists an orthogonal $Q \in \mathbb{R}^{n \times n}$ such that $D = Q^{T}AQ$ is a diagonal matrix.

A related decomposition of $A \in \mathbb{R}^{n \times n}$ is the \emph{Schur factorization} $A = QTQ^{T}$, where $Q \in \mathbb{R}^{n \times n}$ is orthogonal, and $T \in \mathbb{R}^{n \times n}$ is an upper-triangular matrix. $A \in \mathbb{R}^{n \times n}$ is called \emph{Schur factorizable} if there exists an orthogonal $Q \in \mathbb{R}^{n \times n}$ such that $T = Q^{T}AQ$ is an upper-triangular matrix.

\vspace*{20pt}

{\large\textbf{Problem 1}} \hfill [10 + 10 + 10 + 20 = 50]
\begin{enumerate}[A.]
\item Prove that two \emph{similar} matrices have identical eigenvalues. As a corollary, prove that the eigenvalues of a \emph{diagonalizable} matrix $A$ is revealed by its diagonal reduction $D = X^{-1}AX$. Does the \emph{Schur factorization} of a square matrix $A \in \mathbb{R}^{n \times n}$ reveal its eigenvalues? Justify.

\item Is every square matrix $A \in \mathbb{R}^{n \times n}$ unitarily diagonalizable? Provide a necessary and sufficient condition for a square matrix $A \in \mathbb{R}^{n \times n}$ to be unitarily diagonalizable, with justification.

\item Is every square matrix $A \in \mathbb{R}^{n \times n}$ Schur factorizable? Provide a necessary and sufficient condition for a square matrix $A \in \mathbb{R}^{n \times n}$ to be Schur factorizable, with justification.

\item Is it possible to diagonalize or Schur factorize a square matrix $A \in \mathbb{R}^{n \times n}$ through a finite number of \emph{unitary similarity transformation} on $A$? If yes, provide the description of such an algorithm, with justification. If not, provide a justification why it may not be possible.

\end{enumerate}


\end{document}