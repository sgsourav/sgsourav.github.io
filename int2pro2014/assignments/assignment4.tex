\documentclass[11pt,a4paper]{article}
\usepackage[margin=1in]{geometry}

\usepackage{amsmath}
\usepackage{amssymb}
\usepackage{wasysym}
\usepackage[T1]{fontenc}

\setlength{\parskip}{5pt}
\setlength{\parindent}{0pt}

\begin{document}

\pagestyle{empty}

\begin{center}
  \hrule
  \vspace*{10pt}
  
  {\huge Introduction to Programming}

  \vspace*{10pt}

  {\large MTech CS, First Year, Indian Statistical Institute}
  
  \vspace*{10pt}  
  \hrule
  \vspace*{20pt}
  
  \textbf{\Large Assignment 4 : File Handling in C}

  Posted on 6 November 2014 $ \ \vert \ $ Due on 17 November 2014

  \vspace*{10pt}
  
\end{center}



\subsection*{Problem 1}

An \emph{anagram} is a type of word play, where the result of rearranging the letters of a word, using all the original letters exactly once, produces a new word that belongs to some predefined dictionary.

Write a C program that takes an English word as a \emph{command-line input} from the user, and outputs all possible anagrams of the same, using a standard dictionary located at \texttt{/usr/share/dict}.


{\flushleft\textbf{Bonus problem:}} Take an English phrase instead of a single word (e.g., \emph{eleven plus two}), and output anagrammed phrases (e.g., \emph{twelve plus one}) for the same, with the words of the output phrases belonging to the predefined standard dictionary located at \texttt{/usr/share/dict}.


\subsection*{Problem 2}

Directory listing programs like \texttt{ls} on UNIX platforms or \texttt{dir} on DOS platforms provide various information about the files and folders within a target directory. Check \texttt{ls -l} to know more.

Write a C program that takes a valid UNIX path (e.g., \texttt{$\sim$/Desktop/}) as a \emph{command-line input} from the user, and lists useful details of all files and folders located at that path. The output of your program should ideally simulate the output of \texttt{ls -l $\sim$/Desktop/}.


\vfill

\hfill Good Luck! {\Large\smiley{}}

\end{document}