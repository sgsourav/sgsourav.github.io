\documentclass[11pt,a4paper]{article}
\usepackage[margin=1in]{geometry}

\usepackage{amsmath}
\usepackage{amssymb}
\usepackage{wasysym}
\usepackage[T1]{fontenc}

\setlength{\parskip}{5pt}
\setlength{\parindent}{0pt}

\begin{document}

\pagestyle{empty}

\begin{center}
  \hrule
  \vspace*{10pt}
  
  {\huge Introduction to Programming}

  \vspace*{10pt}

  {\large MTech CS, First Year, Indian Statistical Institute}
  
  \vspace*{10pt}  
  \hrule
  \vspace*{20pt}
  
  \textbf{\Large Assignment 2 : Pointers in C}

  Posted on 28 August 2014 $ \ \vert \ $ Due on 28 August 2014

  \vspace*{10pt}
  
\end{center}

\subsubsection*{Submission Policy}

In-class assignment, to be completed within the class time of 3 hours on 28 August 2014.

Submit separate codes for each of the following problems, combined in a compressed folder \texttt{\{RR\}\_assign2\_solutions.zip}, where \texttt{\{RR\}} is your roll number (e.g., \texttt{mtc1421}). You should provide a README portion within each C source file, including the compilation and execution commands, and the input formats that a user should follow to execute the program.


\subsection*{Problem 1}

Write a generic Linear Search program in C, which is capable of searching any generic data-type in an indexed collection of similar data-types. If the input data-type, called the \emph{key}, is found within the indexed collection, the program should return either the index of its location within the collection, or the absolute memory address of the key data-type within the collection.


\subsection*{Problem 2}

Write a generic Sorting program in C, which is capable of sorting an indexed collection of similar but generic data-types. The sorting algorithm may be any standard sorting algorithm of the programmer's choice, but the data-type and the indexed collection will be provided by the user.


\vfill

\hfill Good Luck! {\Large\smiley{}}

\end{document}