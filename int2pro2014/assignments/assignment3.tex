\documentclass[11pt,a4paper]{article}
\usepackage[margin=1in]{geometry}

\usepackage{amsmath}
\usepackage{amssymb}
\usepackage{wasysym}
\usepackage[T1]{fontenc}

\setlength{\parskip}{5pt}
\setlength{\parindent}{0pt}

\begin{document}

\pagestyle{empty}

\begin{center}
  \hrule
  \vspace*{10pt}
  
  {\huge Introduction to Programming}

  \vspace*{10pt}

  {\large MTech CS, First Year, Indian Statistical Institute}
  
  \vspace*{10pt}  
  \hrule
  \vspace*{20pt}
  
  \textbf{\Large Assignment 3 : Basic Data Structure}

  Posted on 1 October 2014 $ \ \vert \ $ Due on 20 October 2014

  \vspace*{10pt}
  
\end{center}



\subsection*{Problem 1}

Write a C code to construct a singly linked list, and to perform basic operations on the same. The program should allow the user to create a new linked list, add a node to the linked list at a location desired by the user (beginning, end, or after a specific node), delete a node from a location desired by the user (beginning, end, or specified otherwise), and print the linked list.


\subsection*{Problem 2}

Given an existing singly linked list (by address of the first node), write a function \texttt{findLoop} to determine if there exists any loop in the list. Write another function \texttt{removeLoop} to break/remove loop(s) in the list, if there exists any. For this problem, provide a theoretical write-up explaining the algorithm you use, and discuss its runtime.


\subsection*{Problem 3}

Implement a stack and a queue, independently, using singly linked lists. The structures should have basic operational functions -- \texttt{stackPush}, \texttt{stackPop}, \texttt{queuePush}, \texttt{queuePop} -- and independent functions \texttt{stackPrint} and \texttt{queuePrint} to print the elements in the respective structures.


\subsection*{Problem 4}

Implement a queue using only stack(s) as components; the standard queue functions \texttt{queuePush} and \texttt{queuePop} should be simulated by calling native stack functions like \texttt{stackPush} and \texttt{stackPop} on the component stack(s). Similarly, implement a stack using only queue(s) as components.


\subsection*{Problem 5}

Write a C code to implement a single-line command-line scientific calculator. The command-line input from the user will be a completely parenthesized \emph{infix} mathematical expression (without gaps) comprising of basic binary arithmetic operations like \texttt{+,-,*,/,\%} and \texttt{\^} (power), and basic mathematical functions like \texttt{sin,cos,tan,log,exp} etc. You may use the \texttt{math.h} library.

\textbf{Bonus:} In addition to evaluating the expression, print the \emph{prefix} and \emph{postfix} forms, and allow the user to input rational numbers in decimal format within the aforesaid mathematical expression. 



\vfill

\hfill Good Luck! {\Large\smiley{}}

\end{document}